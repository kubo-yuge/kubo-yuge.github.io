\documentclass[a4paper,10pt,onecolumn,oneside,notitlepage,final]{jsarticle} % コンパイル:platex
\input{(kubo)package19.txt} % 標準パッケージをまとめた
\input{(kubo)settei19.txt} % 環境の設定をまとめた
\input{(kubo)teigi19.txt}% 個人的定義をまとめた
\setmargin{20}{20}{14}{12} % 左右上下のマージンを再指定する。
%% 用紙を縦方向で使う。2段組にしない。

%% ローカルコマンドの定義
\newcommand{\cbox}[1]{\mbox{\small\fboxsep=2pt\fboxrule=1sp$\fbox{#1}$}}%% 列を示す四角囲み

\begin{document}
%% 表示切替をしない。
%\opentxt%% answersによりxxx環境の内容を一時ファイルに出力する。
%\begin{CKdata}
\large
\noindent 基本変形を用いて行列式を求める。\\ %% タイトル

行列式の性質(ここで使うもの。※ページは,実教出版「新版 線形代数」)\\

p.103 例題3\\
$\left|\begin{array}{ccccc}
a_{11}&a_{12}&\cdots&a_{1n}\\
   0  &a_{22}&\cdots&a_{2n}\\
\vdots&\vdots&\ddots&\vdots\\
   0  &a_{n2}&\cdots&a_{nn}\\
\end{array}\right|
=a_{11}
\left|\begin{array}{cccc}
a_{22}&\cdots&a_{2n}\\
\vdots&\ddots&\vdots\\
a_{n2}&\cdots&a_{nn}\\
\end{array}\right|$\\

p.104\\
正方行列$A$について,$|A|=|^tA|$\\

p.106 {\bf 行列式の性質{[1]}}\\
(I) 2つの行(列)を交換すると,行列式の符号が変化する。\\
(II) 2つの行(列)が等しい行列式の値は$0$になる。\\

{\bf 例題3の発展}$(\ast)$:
p.103 例題3と行(列)の交換を組合わせる。\\
行列$A$の$(i,j)$成分$a_{ij}$について,\\
$i$行目の$a_{ij}$以外の成分がすべて$0$,または,$j$列目の$a_{ij}$以外の成分がすべて$0$であれば,\\
$|A|=(-1)^{i+j}\times a_{ij}\times D_{ij}$\\
※ $D_{ij}$は,$A$の$(i,j)$小行列式(p.113)\\
%% 小行列式について,実教出版「新版 線形代数」p.113,大日本図書「新 線形代数」.97

p.106 {\bf 行列式の性質{[2]}}\\
(III) 2つの行(列)の成分に共通な因数は,行列式の因数としてくくりだせる。\\

p.109 {\bf 行列式の性質{[3]}}\\
(IV) 1つの行(列)の各成分が2数の和として表されるとき,行列式は2つの行列式の和となる。\\
(V) 1つの行(列)を何倍かして,他の行に加えても行列式の値は変化しない。\\


\newpage

計算例:p.109 練習9(1)\\%% 実教出版「新版 線形代数」

%% 出発とする式
$\begin{array}{|rrr@{\quad}|}
\hspace{6mm}&\hspace{6mm}&\hspace{6mm}\\[-7mm]
   1&   3&  -1\\[3mm]
   2&   5&   2\\[3mm]
   0&   4&   1\\
\end{array}$
%% 操作内容
$\begin{array}{ll}
\hfill = \hfill\\
\maru{2}+\maru{1}\times(-2)\\
\end{array}$
%% 得た式
$\begin{array}{|rrr@{\quad}|}
\hspace{6mm}&\hspace{6mm}&\hspace{6mm}\\[-7mm]
   1&   3&  -1\\[3mm]
   0&  -1&   4\\[3mm]
   0&   4&   1\\
\end{array}$
%% 操作内容
$\begin{array}{ll}
%\downarrow &
\hfill = \hfill\\
\text{p.103 例題3}
\end{array}$
%% 得た式
$1\times\begin{array}{|rr@{\quad}|}
\hspace{6mm}&\hspace{6mm}\\[-7mm]
  -1&   4\\[3mm]
   4&   1\\
\end{array}$

%% 求めた行列式
\vspace{5mm}
$=-17$

\vspace{10mm}
計算例:p.109 練習9(2)\\%% 実教出版「新版 線形代数」

%% 出発とする式
$\begin{array}{|rrr@{\quad}|}
\hspace{6mm}&\hspace{6mm}&\hspace{6mm}\\[-7mm]
   2&   0&  -4\\[3mm]
   1&   1&   3\\[3mm]
  -1&   5&   2\\
\end{array}$
%% 操作内容
$\begin{array}{ll}
\hfill = \hfill\\
\cbox{3}+\cbox{1}\times2\\
\end{array}$
%% 得た式
$\begin{array}{|rrr@{\quad}|}
\hspace{6mm}&\hspace{6mm}&\hspace{6mm}\\[-7mm]
   2&   0&   0\\[3mm]
   1&   1&   5\\[3mm]
  -1&   5&   0\\
\end{array}$
%% 操作内容
$\begin{array}{ll}
%\downarrow &
\hfill = \hfill\\
\text{p.103 例題3}
\end{array}$
%% 得た式
$2\times\begin{array}{|rr@{\quad}|}
\hspace{6mm}&\hspace{6mm}\\[-7mm]
   1&   5\\[3mm]
   5&   0\\
\end{array}$

%% 求めた行列式
\vspace{5mm}
$=2\times(-25)=-50$
\vspace{10mm}


計算例:p.110 練習10(1)\\%% 実教出版「新版 線形代数」

%% 出発とする式
$\begin{array}{|rrr@{\quad}|}
\hspace{6mm}&\hspace{6mm}&\hspace{6mm}\\[-7mm]
   1&   3&   2\\[3mm]
   2&   5&   4\\[3mm]
   3&   6&   7\\
\end{array}$
%% 操作内容
$\begin{array}{ll}
\hfill = \hfill\\
\maru{2}+\maru{1}\times(-2)\\
\maru{3}+\maru{1}\times(-3)\\
\end{array}$
%% 得た式
$\begin{array}{|rrr@{\quad}|}
\hspace{6mm}&\hspace{6mm}&\hspace{6mm}\\[-7mm]
   1&   3&   2\\[3mm]
   0&  -1&   0\\[3mm]
   0&  -3&   1\\
\end{array}$
%% 操作内容
$\begin{array}{ll}
%\downarrow &
\hfill = \hfill\\
\text{p.103 例題3}
\end{array}$
%% 得た式
$1\times\begin{array}{|rr@{\quad}|}
\hspace{6mm}&\hspace{6mm}\\[-7mm]
  -1&   0\\[3mm]
  -3&   1\\
\end{array}$

%% 求めた行列式
\vspace{5mm}
$=-1$
\vspace{10mm}%% 一つの計算例の終わり


計算例:p.110 練習10(2)\\%% 実教出版「新版 線形代数」

%% 出発とする式
$\begin{array}{|rrrr@{\quad}|}
\hspace{6mm}&\hspace{6mm}&\hspace{6mm}&\hspace{6mm}\\[-7mm]
   1&   1&   1&   1\\[3mm]
   1&   2&   2&   2\\[3mm]
   1&   2&   3&   3\\[3mm]
   1&   2&   3&   6\\
\end{array}$
%% 操作内容
$\begin{array}{ll}
\\ \\ \hfill = \hfill\\
\maru{2}+\maru{1}\times(-1)\\
\maru{3}+\maru{1}\times(-1)\\
\maru{4}+\maru{1}\times(-1)\\
\end{array}$
%% 得た式
$\begin{array}{|rrrr@{\quad}|}
\hspace{6mm}&\hspace{6mm}&\hspace{6mm}&\hspace{6mm}\\[-7mm]
   1&   1&   1&   1\\[3mm]
   0&   1&   1&   1\\[3mm]
   0&   1&   2&   2\\[3mm]
   0&   1&   2&   5\\
\end{array}$
%% 操作内容
$\begin{array}{ll}
%\downarrow &
\hfill = \hfill\\
\text{p.103 例題3}
\end{array}$
%% 得た式
$\begin{array}{|rrr@{\quad}|}
\hspace{6mm}&\hspace{6mm}&\hspace{6mm}\\[-7mm]
   1&   1&   1\\[3mm]
   1&   2&   2\\[3mm]
   1&   2&   5\\
\end{array}$

%% 操作内容
$\begin{array}{ll}
\\ \\ \hfill = \hfill\\
\maru{2}+\maru{1}\times(-1)\\
\maru{3}+\maru{1}\times(-1)\\
\end{array}$
%% 得た式
$\begin{array}{|rrr@{\quad}|}
\hspace{6mm}&\hspace{6mm}&\hspace{6mm}\\[-7mm]
   1&   1&   1\\[3mm]
   0&   1&   1\\[3mm]
   0&   1&   4\\
\end{array}$
%% 操作内容
$\begin{array}{ll}
%\downarrow &
\hfill = \hfill\\
\text{p.103 例題3}
\end{array}$
%% 得た式
$\begin{array}{|rr@{\quad}|}
\hspace{6mm}&\hspace{6mm}\\[-7mm]
   1&   1\\[3mm]
   1&   4\\
\end{array}$
%% 求めた行列式
\vspace{5mm}
$=3$
\vspace{10mm}%% 一つの計算例の終わり

\newpage

計算例:p.110 練習10(3)\\%% 実教出版「新版 線形代数」

%% 出発とする式
$\begin{array}{|rrrr@{\quad}|}
\hspace{6mm}&\hspace{6mm}&\hspace{6mm}&\hspace{6mm}\\[-7mm]
   2&   3&   6&   1\\[3mm]
  -1&   1&  -4&  -3\\[3mm]
   0&   2&   5&  -2\\[3mm]
   3&   1&   4&   0\\
\end{array}$
%% 操作内容
$\begin{array}{ll}
\hfill = \hfill\\
\maru{1}+\maru{2}\times 2\\
\maru{4}+\maru{2}\times 3\\
\end{array}$
%% 得た式
$\begin{array}{|rrrr@{\quad}|}
\hspace{6mm}&\hspace{6mm}&\hspace{6mm}&\hspace{6mm}\\[-7mm]
   0&   5&  -2&  -5\\[3mm]
  -1&   1&  -4&  -3\\[3mm]
   0&   2&   5&  -2\\[3mm]
   0&   4&  -8&  -9\\
\end{array}$

%% 操作内容
$\begin{array}{ll}
\\ \\ \hfill = \hfill\\
\maru{1},\maru{2}\\ の交換\\
\end{array}$
%% 得た式
$-\ \begin{array}{|rrrr@{\quad}|}
\hspace{6mm}&\hspace{6mm}&\hspace{6mm}&\hspace{6mm}\\[-7mm]
  -1&   1&  -4&  -3\\[3mm]
   0&   5&  -2&  -5\\[3mm]
   0&   2&   5&  -2\\[3mm]
   0&   4&  -8&  -9\\
\end{array}$
%% 操作内容
$\begin{array}{ll}
%\downarrow &
\hfill = \hfill\\
\text{p.103 例題3}
\end{array}$
%% 得た式
$-(-1)\times \begin{array}{|rrr@{\quad}|}
\hspace{6mm}&\hspace{6mm}&\hspace{6mm}\\[-7mm]
   5&  -2&  -5\\[3mm]
   2&   5&  -2\\[3mm]
   4&  -8&  -9\\
\end{array}$

%% 操作内容
$\begin{array}{ll}
\hfill = \hfill\\
\cbox{1}+\cbox{3}\\
\end{array}$
%% 得た式
$\begin{array}{|rrr@{\quad}|}
\hspace{6mm}&\hspace{6mm}&\hspace{6mm}\\[-7mm]
   0&  -2&  -5\\[3mm]
   0&   5&  -2\\[3mm]
  -5&  -8&  -9\\
\end{array}$
%% 操作内容
$\begin{array}{ll}
%\downarrow &
\hfill = \hfill\\
\text{例題3}\\
\hfill の拡張(\ast) \hfill
\end{array}$
%% 得た式
$(-1)^2\times (-5)\ \begin{array}{|rr@{\quad}|}
\hspace{6mm}&\hspace{6mm}\\[-7mm]
  -2&  -5\\[3mm]
   5&  -2\\
\end{array}$
%% 求めた行列式
\vspace{5mm}
$=-5\times(4+25)=-145$

$(\ast)$について\\
$3$行目を順送りに上の行と交換して$1$行目まで移動してから,例題3を適用する。



\vspace{10mm}
計算例:p.110 練習10(4)\\%% 実教出版「新版 線形代数」

%% 出発とする式
$\begin{array}{|rrrr@{\quad}|}
\hspace{6mm}&\hspace{6mm}&\hspace{6mm}&\hspace{6mm}\\[-7mm]
   3&   0&   1&   6\\[3mm]
   1&   2&   2&  -1\\[3mm]
   2&  -1&   5&   0\\[3mm]
   1&   4&   1&   1\\
\end{array}$
%% 操作内容
$\begin{array}{ll}
\hfill = \hfill\\
\maru{1}+\maru{2}\times 6\\
\maru{4}+\maru{2}\times 1\\
\end{array}$
%% 得た式
$\begin{array}{|rrrr@{\quad}|}
\hspace{6mm}&\hspace{6mm}&\hspace{6mm}&\hspace{6mm}\\[-7mm]
   9&  12&  13&   0\\[3mm]
   1&   2&   2&  -1\\[3mm]
   2&  -1&   5&   0\\[3mm]
   2&   6&   3&   0\\
\end{array}$

%% 操作内容
$\begin{array}{ll}
行・列の\\
\hfill = \hfill\\
交換と例題3\\
\end{array}$
%% 得た式
$(-1)^{2+4}\times(-1)\ \begin{array}{|rrr@{\quad}|}
\hspace{6mm}&\hspace{6mm}&\hspace{6mm}\\[-7mm]
   9&  12&  13\\[3mm]
   2&  -1&   5\\[3mm]
   2&   6&   3\\
\end{array}$
%% 操作内容
$\begin{array}{ll}
\\ \\ \hfill = \hfill\\
\maru{1}+\maru{3}\times(-2)\\
\end{array}$
%% 得た式
$(-1)\ \begin{array}{|rrr@{\quad}|}
\hspace{6mm}&\hspace{6mm}&\hspace{6mm}\\[-7mm]
   5&   0&   7\\[3mm]
   2&  -1&   5\\[3mm]
   2&   6&   3\\
\end{array}$
%% 操作内容
$\begin{array}{ll}
\\ \\ \hfill = \hfill\\
\maru{3}+\maru{2}\times 6\\
\end{array}$
%% 得た式
$(-1)\ \begin{array}{|rrr@{\quad}|}
\hspace{6mm}&\hspace{6mm}&\hspace{6mm}\\[-7mm]
   5&   0&   7\\[3mm]
   2&  -1&   5\\[3mm]
  14&   0&  33\\
\end{array}$

\vspace{4mm}
\noindent %% 操作内容
$\begin{array}{ll}
%\downarrow &
(2,2)成分\\[-2mm]
\hfill = \hfill\\[-2mm]
を外へ
\end{array}$
%% 得た式
$(-1)\times(-1)^{2+2}\times(-1)\begin{array}{|rr@{\quad}|}
\hspace{6mm}&\hspace{6mm}\\[-7mm]
   5&   7\\[3mm]
  14&  33\\
\end{array}$
$=5\times 33-7\times 14=67$

\vspace{3mm}
2行目の成分が大きい値なのを避けるなら,
\vspace{3mm}

%% 操作内容
$\begin{array}{ll}
\hfill = \hfill\\
\maru{3}+\maru{1}\times(-3)\\
\end{array}$
%% 得た式
$\begin{array}{|rr@{\quad}|}
\hspace{6mm}&\hspace{6mm}\\[-7mm]
   5&   7\\[3mm]
  -1&  12\\
\end{array}$
%% 求めた行列式
\vspace{5mm}
$=5\times 12-(-1)\times 7=67$

\newpage

%\end{CKdata}
%\closetxt%% ここまでの内容をファイルに出力する

%\renewcommand{\ctoi}{Ckeep} % ctoi部分を通常の黒色
%\renewcommand{\ckai}{Cmagenta} % ckai部分をマゼンタ色
%\printdoc % 上で定義した本文の挿入

\end{document}